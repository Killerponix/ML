\documentclass[ngerman]{article}
\usepackage[margin=1cm,landscape,a1paper]{geometry}
\usepackage[utf8]{inputenc}
\usepackage{graphicx}
\usepackage{tikz}
\usepackage{trfsigns}
\usepackage{pgfplots}
\usepackage[landscape]{geometry}
\usetikzlibrary{patterns}
\usetikzlibrary{shapes,arrows,shadows,positioning, calc, matrix, decorations.pathreplacing,decorations.markings}
\usetikzlibrary{shapes,snakes}
\usetikzlibrary{decorations.pathreplacing}
\usetikzlibrary{shapes.geometric}
\usetikzlibrary{plotmarks, shapes,arrows,shadows,positioning, calc, shapes.multipart, automata}
\usetikzlibrary{positioning,shapes.multipart,calc,arrows.spaced,graphs,arrows.meta}
\usetikzlibrary{angles,quotes}
\usetikzlibrary{backgrounds}
\usepackage{tikz-3dplot}
\usepackage{csvsimple}
\usepackage{pgfplots}
\usepackage[locale=DE, per=frac, fraction=frac]{siunitx}
\usetikzlibrary{calc}
\pgfplotsset{compat=1.10}
\usepgfplotslibrary{patchplots}

\begin{document}
    \begin{figure}[h!]
        \begin{tikzpicture}[sibling distance=10em,   %10
            level 1/.style = {sibling distance=40cm}, %6
            level 2/.style = {sibling distance=30cm},  %3
            level 3/.style = {sibling distance=13cm},  %2
            level 4/.style = {sibling distance=6cm},  %2
            level distance = 1cm,
            every node/.style = {shape=rectangle, rounded corners,
                draw, align=center}]]

%ab Hier Output Methode: print(myTree.bTree.print(6))

           \node {0 || x[0] = 1 } child { node {4 || x[2] = 1 } child { node[fill=gray!30] {5 || 0 }  } child { node {6 || x[3] = 1 } child { node[fill=gray!30] {7 || 0 }  } child { node[fill=gray!30] {8 || 1 }  }  }  } child { node {1 || x[1] = 1 } child { node[fill=gray!30] {3 || 1 }  } child { node[fill=gray!30] {2 || 0 }  }  }  ;

%Ende
 
        \end{tikzpicture}
        \caption{\label{fig:EinBaumRadfahren:2} Baum Aufgabe3 Übungsblatt4}
    \end{figure}
    
\end{document}
